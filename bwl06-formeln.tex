\documentclass[a4paper,12pt]{scrartcl} 

%\usepackage[latin1]{inputenc} 
%Apple \usepackage[applemac]{inputenc} 
\usepackage[utf8]{inputenc}
\usepackage[ngerman]{babel}
\usepackage[T1]{fontenc}

%Das Paket erzeugt ein anklickbares Verzeichnis in der PDF-Datei.
\usepackage[hyperfootnotes=false,colorlinks=true,linkcolor=black,urlcolor=black]{hyperref}

%Das Paket wird fr die anderthalb-zeiligen Zeilenabstand bentigt
\usepackage{setspace}

%Einrückung eines neuen Absatzes
\setlength{\parindent}{0em}

%Definition der Rnder
\usepackage[paper=a4paper,left=30mm,right=30mm,top=30mm,bottom=30mm]{geometry} 

\usepackage{amsfonts}
\usepackage{amsmath}
\usepackage{cancel}
\usepackage{graphicx}
\usepackage{mathcomp}
\usepackage{polynom}
\usepackage{hyperref}
\usepackage[nottoc]{tocbibind} % Anzeigen des Verzeichnisse im TOC
\usepackage{mathtools} 

%Abstand der Fußnoten
\deffootnote{1em}{1em}{\textsuperscript{\thefootnotemark\ }}

%Regeln, bis zu welcher Tiefe (section,subsection,subsubsection) Überschriften angezeigt werden sollen (Anzeige der Überschriften im Verzeichnis / Anzeige der Nummerierung)
\setcounter{tocdepth}{3}
\setcounter{secnumdepth}{3}
\pagenumbering{roman}

%-------------------
%Ende des Kopfbereiches
%-------------------


\begin{document}



%Beginn der Titelseite
\begin{titlepage}
\begin{small}
\vfill {AKAD\\ 
Bachelor of Science (Wirtschaftsinformatik) \\ 
Modulzusammenfassung}
\end{small}


\begin{center}
\begin{Large}
\vfill {\textsf{\textbf{
BWL06 \\
\vspace*{1cm} 
Formelsammlung
}}}
\end{Large}
\end{center}

\begin{small}
\vfill Daniel Falkner \\ Rotbach 529 \\  94078 Freyung \\  daniel.falkner@akad.de \\ 
\today
\end{small}

\end{titlepage}
%Ende der Titelseite


%Inhaltsverzeichnis (aktualisiert sich erst nach dem zweiten Setzen)
\tableofcontents

%Beginn einer neuen Seite
\clearpage

%Anderthalbzeiliger Zeilenabstand ab hier
\onehalfspacing

\pagestyle{plain}


\pagenumbering{arabic}

\section{Investitionsrechnung bei sicheren Erwartungen - statische Verfahren}
\subsection{Kostenvergleichsrechnung}
\textit{Wähle diejenige Investition mit den geringsten durchschnittlichen Gesamtkosten.}
\subsubsection{gebundenes Kapital}
$\oslash$ gebundenes Kapital = $\cfrac{Anschaffungswert + Restwert}{2}$
\subsubsection{kalkulatorische Zinsen}
$\oslash$ kalkulatorische Zinsen = Kalkulationszinssatz * $\oslash$ gebundenes Kapital
\subsubsection{Kalkulatorische Abschreibungen}
Kalkulatorische Abschreibungen = $\cfrac{Anschaffungswert - Restwert}{Nutzungsdauer}$
\subsubsection{Kosten}
$K = K_f + K_v$
\subsubsection{variable Stückkosten}
$k_v = \cfrac{K-K_f}{x}$

\subsection{Gewinnvergleichsrechnung}
\textit{Wähle diejenige Alternative mit dem höchsten (durchschnittlichen Gewinn).}
\subsubsection{Gewinn}
Gewinn = Erlös - Kosten

\subsection{Rentabilitätsvergleichsrechnung}
\textit{Realisiere jede Investition, die eine geforderte Mindestrentabilität erwirtschaftet.}
\subsubsection{Rentabilität}
Rentabilität = $\cfrac{\oslash Gewinn + \oslash kalkulatorische Zinsen}{\oslash gebundenes Kaputal} * 100\% $

\subsection{Amortisationsrechnung}
\textit{Realisiere Investitionen, soweit ihre Amortisationsdauer geringer ist als eine maximal zulässige (subjektiv vorgegebene) Dauer }\footnote{Diese Dauer wird in Abhängigkeit von der betrachteten Investition (Risiko) und der Liquiditätslage des Unternehmens festgelegt}
\subsubsection{Amortisationsdauer}
Amortisationsdauer = $\cfrac{Urpr"unglich~eingesetztes~Kapital}{j"ahrliche~Kapitalwiedergewinnung~aus~Zahlungsübersch"ussen}$
\subsubsection{Amortisationsdauer Durchschnittsmethode}
Amortisationsdauer = $\cfrac{Urpr"unglich eingesetztes Kapital}{\oslash Gewinn + \oslash kalkulatorische EK-Zinsen + \oslash Abschreibungen}$
\subsubsection{Amortisationsdauer Kummulationsmethode}
Amortisationsdauer = Anzahl der Jahre vor der vollständigen Amortisation + \\ 
\begin{small}
$\cfrac{zur~Amortisation~fehlender~Betrag \\ am~Ende~der~letzten~Periode ohne~vollst"andige~Amortisation}{Nettozahlung~im~Jahr~der~Amortisation}$
\end{small}



Formal $n^* = (n^+ - 1) + \cfrac{A_0 + \displaystyle\sum_{t=1}^{n^+-1} E-A}{(E - A)_{n^+}}$

\section{Investitionsrechnung bei sicheren Erwartungen - dynamische Verfahren}
\subsection{Anforderungen an einen plausiblen Zinssatz}
\subsubsection{gewogener Kapitalkostensatz WACC}
gewogener Kapitalkostensatz WACC \footnote{WACC = weighted average cost of capital} = $\cfrac{EK * i_{EK} + FK * i_{FK}}{EK + FK}$ \\
\vspace*{5mm}

\begin{small}
\textit{mit EK = Eigenkapital, FK = Fremdkapital, \\ $i_{EK}$ = Zinssatz für das EK (=Mindestrenditeforderung des Eigenkapitalgebers) \\ $i_{FK}$ = Zinssatz für das FK (=Kosten des Fremdkapitals)
}
\end{small}
\subsubsection{Eigenkapitalkosten}
Eigenkapitalkosten = risikoloser Zins + Risikoprämie

\subsection{Abzinsungsfaktor}
$(1+i)^{-n}$

\subsection{dynamische Amortisationsrechnung}
Amortisationsdauer = Anzahl der Jahre vor der vollständigen Amortisation \footnote{aus auf den Zeitpunkt $t_0$ diskontierten Zahlungen} + \\ 
\begin{small}
$\cfrac{Summe~der~auf~den~Zeitpunkt~t_0~ diskontierten~Zahlungen~bis~zur~lezten~Periode~ohne~vollständige~Amortisation}{auf~t_0~diskontierte~Nettozahlung~im~Jahr~der~Amortisation}$
\end{small}

\subsection{Kapitalwertmethode}
\subsubsection{Ermittlung von Kapitalwerten}
$C_0 = \displaystyle\sum_{t=1}^{n} (E_t - A_t) * (1+i)^{-1}$ bzw. \\
$C_0 = -A_0 + \displaystyle\sum_{t=1}^{n} (E_t - A_t) * (1+i)^{-1}$ \\
$C_0 = \displaystyle\sum_{t=1}^{n} \cfrac{Z_t}{(1+i)^t}$ \\
$C_0 = Z_0 + \displaystyle\sum_{t=1}^{n} \cfrac{Z_t}{(1+i)^t}$ 
\\
\vspace*{5mm}

C > 0 Ertrag der Investition "ubersteigt den Kalkulationszinssatz = Verm"ogenszuwachs bezogen auf $t_0$ \\
C < 0 Ertrag der Investition liegt unter dem Kalkulationszinssatz = Verm"ogensabnahme bezogen auf $t_0$
\vspace*{5mm}

Mit Restwert \footnote{Sofern der Restwert nicht schon bei der Ermittlung der letzten Nettozahlung mit einbezogen wurde.} :
$C_0 = Z_0 + \displaystyle\sum_{t=1}^{n} \cfrac{Z_t}{(1+i)^t} + \cfrac{RW_n}{(1+i)^n}$  \\

\subsection{Annuitätenmethode}
\subsubsection{Annuitätenfaktor}
$\cfrac{i * (1+1)^n}{(1+i)^n -1}$
\subsubsection{Annuität Gesamtformel}
$ g = C_0 * \cfrac{i * (1+1)^n}{(1+i)^n -1}$





\end{document}


